%%%%%%%%%%%%%%%%%%%%%%%%%%%%%%%%%%%%%%%%%
% Prácticas Sistemas Inteligentes
%
% Author:
% Hugo Sánchez
%%%%%%%%%%%%%%%%%%%%%%%%%%%%%%%%%%%%%%%%%

\documentclass[a4paper,11pt, includehead]{article}
%------------------------------
%   PACKAGES AND DOCUMENT CONFIG
%------------------------------

\usepackage[top=1.5cm,bottom=2.5cm,left=2cm,right=2cm]{geometry} % margins
\usepackage[spanish,es-nodecimaldot]{babel} % spanish
\usepackage[utf8]{inputenc}
\usepackage[T1]{fontenc}
\usepackage{amssymb,amsmath,amsthm} % math
\usepackage{tabularx, multirow} % tables
\usepackage{tikz} % pictures
\usetikzlibrary{calc, arrows.meta, positioning, shapes.multipart, shapes.geometric, shapes.arrows}
\usepackage{graphicx, xcolor} % images and colours
\usepackage{float}
\usepackage{fancyhdr} % headers
\usepackage[bottom]{footmisc} % footers to bottom
\usepackage[ruled]{algorithm2e} % algorithms
\usepackage{wrapfig}
\usepackage{colortbl}
\usepackage{booktabs} % better tables
\usepackage{listings} % file listings
\usepackage{url, cite} % references
\usepackage{enumitem}
\usepackage{tcolorbox}
\usepackage{pmboxdraw}
\usepackage[framemethod=tikz]{mdframed} % Allows defining custom boxed/framed environments
\usepackage[colorlinks=true, linkcolor=blue, citecolor=red, urlcolor=blue]{hyperref}

\hypersetup{
	pdftitle={Sistemas Basados en Reglas},
	pdfsubject={SSII},
	pdfauthor={Hugo Sánchez Martínez}
}

%------------------------------
%	COMMANDS
%------------------------------

\newcommand\university{Universidad de Murcia}
\newcommand\subject{Sistemas Inteligentes}

\newcommand{\pictures}{pictures/}
\newcommand{\pruebas}{../pruebas/}

\renewcommand{\headrulewidth}{0pt} % no header rule


% change ToC title
\addto\captionsspanish{ 
	\renewcommand{\contentsname}
	{Contenidos}
}

% colores
\colorlet{darkgreen}{green!75!black}
\colorlet{darkred}{red!75!black}

\lstset{
	basicstyle=\ttfamily, % Typeset listings in monospace font
}

\definecolor{mGreen}{rgb}{0,0.6,0}
\definecolor{mGray}{rgb}{0.5,0.5,0.5}
\definecolor{mPurple}{rgb}{0.58,0,0.82}
\definecolor{backgroundColour}{rgb}{0.95,0.95,0.92}

\lstdefinestyle{CStyle}{
	backgroundcolor=\color{backgroundColour},
	commentstyle=\color{mGreen},
	keywordstyle=\color{magenta},
	numberstyle=\tiny\color{mGray},
	stringstyle=\color{mPurple},
	basicstyle=\footnotesize,
	captionpos=b,
	breakatwhitespace=false,
	keepspaces=true,
	numbers=left,
	numbersep=5pt,
	showspaces=false,
	showstringspaces=false,
	showtabs=false,
	tabsize=2,
	language=C++
}

\tcbset{
	simplecmd/.style={
		colback=gray!10,
		colframe=black,
		boxrule=0.25mm,
		rounded corners,
		fontupper=\ttfamily,
		before=\medskip,
		after=\medskip,
	}
}


\mdfdefinestyle{warning}{
	topline=false, bottomline=false,
	leftline=false, rightline=false,
	nobreak,
	singleextra={%
		\draw(P-|O)++(-0.5em,0)node(tmp1){};
		\draw(P-|O)++(0.5em,0)node(tmp2){};
		\fill[black,rotate around={45:(P-|O)}](tmp1)rectangle(tmp2);
		\node at(P-|O){\color{white}\scriptsize\bf !};
		\draw[very thick](P-|O)++(0,-1em)--(O);%--(O-|P);
	}
} % import style and config
\renewcommand{\headrulewidth}{0.4pt} % no header rule

\title{
	\vspace{-5ex}
	{\large \textsc{\subject}}\\[1ex]
	\hrule
	\vspace{3ex}
	{\huge \textsc{\underline{Sistemas Basados en Reglas:}}}\\
	{\huge \textsc{\underline{Manual de Usuario}}}\\[2ex]
	{\small Hugo Sánchez Martínez\\[-2ex]
	\texttt{hugo.s.m@um.es}} % Author and email address
	
	\vspace{2ex}
	\hrule
	\vspace{2ex}
	{\normalsize \textsc{Facultad de Informática - Universidad de Murcia}\\[-1ex] Diciembre 2024}
	}

\date{} % University, school and/or department name(s) and a date

%----------------------------------------------------------------------------------------

\begin{document}
	
\maketitle % Print the title

\pagestyle{fancy}

\fancyhf{} % clear existing header/footer
\fancyhead[L]{Hugo Sánchez Martínez}
\fancyhead[R]{SBR: Cuestiones}
\fancyfoot[R]{\thepage}
\setlength{\headsep}{4ex}

\setcounter{page}{1}

\vspace{-3ex}

\noindent {\LARGE \textbf{1.- Estructura}}\\

\noindent El proyecto sigue la siguiente estructura de directorios y ficheros:

\begin{verbatim}
  sbr-fc/
  ├── bin/            # Ejecutable
  ├── src/            # Codigo fuente del proyecto
  ├── include/        # Archivos de cabecera adicionales
  ├── pruebas/        # Pruebas de la práctica
  ├── main.cpp        # Archivo principal
  └── docs/           # Documentación del proyecto
\end{verbatim}

Los archivos necesarios para la generación del ejecutable se encuentran en los directorios \texttt{src/}, \texttt{include/} y el fichero \texttt{main.cpp}.

\vspace{4ex}

\noindent {\LARGE \textbf{2.- Compilación}}\\

Cumpliendo los los requisitos de la práctica, el proyecto está realizado en torno al \textbf{IDE Codeblocks} (17.12). Entre los archivos, se incluye un fichero \textit{Code Blocks Proyect} (\texttt{.cbp}), para importar al IDE el proyecto. Para generar el ejecutable, se recomienda usar la opción \textbf{Rebuild} de Code Blocks. El directorio de destino por defecto es \texttt{bin/}.

\begin{mdframed}[style=warning]
	En algunos casos, será necesario añadir el directorio \texttt{include/} a la lista de directorios en los que debe \textbf{buscar el compilador}. Pasos: Proyecto >\ Build Options >\ Search directories >\ Añadir la carpeta \texttt{include}.
\end{mdframed}

\vspace{4ex}

\noindent {\LARGE \textbf{3.- Ejecución}}\\

\noindent Para usar el ejecutable, ejecutamos el siguiente comando:

\begin{verbatim}
          > sbr-fc.exe -bc [BASE CONOCIMIENTO] -bh [BASE HECHOS]
\end{verbatim}

Donde \texttt{BASE CONOCIMIENTO} y \texttt{BASE HECHOS} son los \textbf{ficheros de la prueba} a ejecutar. Una copia de todos estos ficheros \textbf{se ha incluido en el directorio} destino del ejecutable para simplificar la ejecución.


\end{document}

